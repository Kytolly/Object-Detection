\subsection{开源项目部署与模型下载}
从GitHub克隆所选算法的官方开源项目到本地。
进入每个项目的根目录,根据其`requirements.txt`或`setup.py`安装特有的依赖库。
从官方提供的模型库下载对应算法的预训练权重文件。
例如,YOLOv8的`yolov8n.pt`,MMDetection中Faster R-CNN或FCOS的`.pth`文件。

\subsection{图像数据准备}
在不同校园内不同地点、不同光照条件下拍摄多张图像,
地点包括办公室,食堂,教室等,
时间主要包括上午和晚上的情况以模拟不同情况下的光照条件。
将所有拍摄的图像统一存放在一个新建的文件夹中。

\subsection{目标检测API调用与结果获取}
针对每种选定的算法,查阅模型的官方文档,
编写独立的Python模块,用于加载模型、读取图像、执行推理并获取检测结果。
将每张图像的检测结果直接在图像上绘制的方式保存。

\subsection{结果可视化与分析}
在detectionAPIs.ipynb笔记本上开辟独立的单元格,
执行上一步写好的API调用函数,
分别查看不同算法在相同图像上的检测效果,
定性观察其在目标识别准确性、定位精度、小目标检测、密集目标处理、抗遮挡能力等方面的表现。