\subsection{实现Tiny+KNN方法}
	\begin{enumerate}
		\item 实现student.py/get_tiny_images(image_paths)函数: 
		根据Tiny Images和KNN的实验原理,为输入的图像路径列表生成 Tiny Images 特征。
		输出一个 n x d的 NumPy数组,
		其中 n是图像数量,
		d是 Tiny Images 特征向量的长度(例如 16*16=256);

		\item 完善 `nearest_neighbor_classify(train_image_feats, train_labels, test_image_feats)` 函数
		使用 KNN 分类器对测试图像进行分类。
		为输入的图像路径列表生成 Tiny Images 特征。
		输出一个 n x d的 NumPy数组,
		其中 n是图像数量,
		d是 Tiny Images 特征向量的长度(例如 16*16=256);
		输出一个m x 1的 NumPy 数组,
		其中m是测试图像数量,每个元素是对应的预测类别标签。

	\end{enumerate}


\subsection{实现 Bag of Words (SIFT) + SVM 方法}
	\begin{enumerate}
		\item 实现student.py/build_vocabulary(image_paths, vocab_size)函数
		根据Bag of Words (SIFT) + SVM的实验原理,
		从训练图像中提取 HOG 特征并构建视觉词汇,
		输出一个vocab_size x (z*z*9)的 NumPy 数组,
		代表聚类中心(即视觉词汇)。
		通常会将这个词汇保存到文件vocab.npy。

		\item 实现 get_bags_of_words(image_paths)函数,
		为输入的图像路径列表生成 Bag of Words 直方图特征,
		加载已构建的视觉词汇 (vocab.npy),
		输出一个n x vocab_size的 NumPy 数组,
		其中n是图像数量,vocab_size是词汇大小,
		每个元素是对应的 Bag of Words 直方图特征向量。

		\item 完善 svm_classify(train_image_feats, train_labels, test_image_feats)函数;
		使用 SVM 分类器对测试图像进行分类
		输出一个m x 1的 NumPy 数组,
		其中m是测试图像数量,
		每个元素是对应的预测类别标签。
	\end{enumerate}


\subsection{实现 DNN 网络}
在proj4.ipynb重新开一个单元格,用于实现DNN

\subsection{比较分析}
	\begin{enumerate}
		\item 在完成以上实现后,运行proj4.ipynb所有单元格。
		\item 计算 Tiny+KNN、BoW+SVM 和 DNN在测试集上的分类精度。
		\item 比较这三种方法的精度,分析不同特征表示和不同分类器对场景分类性能的影响。
	\end{enumerate}
    