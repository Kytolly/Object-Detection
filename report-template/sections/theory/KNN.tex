\section{K近邻分类器实验原理}

K近邻(K-Nearest Neighbors, KNN)是一种简单直观的监督学习算法,
主要用于分类和回归任务。
在本实验中,它被用作分类器,特别是与Tiny Images特征结合使用。

\subsection{基本原理}
KNN的核心思想是:一个样本的类别由其在特征空间中的K个最近邻居的类别决定。

\subsection{KNN特点}
\begin{itemize}
    \item \textbf{简单易懂}:原理直观,易于实现
    \item \textbf{无需训练}:仅存储训练数据,无复杂训练过程
    \item \textbf{计算开销}:预测阶段需计算与所有训练样本的距离,大数据集上耗时
    \item \textbf{对噪声敏感}:K值较小时对噪声和离群点敏感
    \item \textbf{维数灾难}:高维特征空间中距离计算意义减弱,数据稀疏
    \item \textbf{非参数模型}:不对数据分布做假设
\end{itemize}

\subsection{算法步骤}
算法流程如下伪代码所示
\documentclass{article}
\usepackage{amsmath}
\usepackage{amssymb}
\usepackage{algorithmicx}
\usepackage{algpseudocode} % For pseudocode formatting
\usepackage{enumitem} % For list formatting

\title{K-Nearest Neighbors (KNN) Classification Pseudocode}
\author{}
\date{}

\begin{document}

\maketitle

\begin{abstract}
the pseudocode for the K-Nearest Neighbors (KNN) classification algorithm.
\end{abstract}

\section{Algorithm: KNN\_Classify}

\begin{algorithmic}[1]
\Procedure{KNN\_Classify}{TestData, TrainingData, TrainingLabels, K}
  \Comment{TestData: Feature vector of a single test sample $x_{test}$}
  \Comment{TrainingData: Set of $n$ training sample feature vectors $\{x_1, x_2, ..., x_n\}$}
  \Comment{TrainingLabels: Corresponding labels $\{y_1, y_2, ..., y_n\}$}
  \Comment{K: Number of nearest neighbors (positive integer)}
  \Comment{Output: Predicted class label $y_{pred}$ for $x_{test}$}

  \State Initialize an empty list: $NeighborsList$
  \Comment{Stores (Distance, Label) tuples}

  \For{each training sample $x_i$ in TrainingData (from $i = 1$ to $n$)}
    \State Calculate the distance between $x_{test}$ and $x_i$:
    \State $Distance_i = \text{Distance}(x_{test}, x_i)$ \Comment{e.g., Euclidean Distance}
    \State Get the corresponding label $y_i$ from TrainingLabels
    \State Add tuple $(Distance_i, y_i)$ to $NeighborsList$
  \EndFor

  \State Sort $NeighborsList$ based on the distance in ascending order
  \Comment{Sorts tuples by their first element}

  \State Select the first K elements from the sorted $NeighborsList$
  \State $K\_Nearest\_Neighbors = NeighborsList[1 \dots K]$ \Comment{These are the K closest neighbors}

  \State Initialize an empty dictionary or map: $LabelCounts$
  \Comment{To count the frequency of labels}

  \For{each neighbor $(distance, label)$ in $K\_Nearest\_Neighbors$}
    \State Increment count for $label$ in $LabelCounts$
    \Comment{e.g., $LabelCounts[label] = LabelCounts.get(label, 0) + 1$}
  \EndFor

  \State Find the label with the maximum count in $LabelCounts$
  \State $y_{pred} = \text{Label with highest count in } LabelCounts$
  \Comment{Handle ties if necessary}

  \State \Return $y_{pred}$

\EndProcedure
\end{algorithmic}

\section{Distance Calculation (Example: Euclidean Distance)}

The Euclidean distance between two feature vectors $\mathbf{x} = (x_1, x_2, \dots, x_D)$ and $\mathbf{y} = (y_1, y_2, \dots, y_D)$ in a D-dimensional space is given by:
\[
d(\mathbf{x}, \mathbf{y}) = \sqrt{\sum_{j=1}^{D} (x_j - y_j)^2}
\]

\end{document}

\subsection{K值选择}
参数K是KNN算法最重要的决定因素:
\begin{itemize}
    \item K=1时,新样本类别由最近训练样本决定,对噪声敏感
    \item 增大K可减少噪声影响,使决策边界更平滑
    \item K过大可能包含不相关邻居,导致性能下降
    \item 通常通过交叉验证选择最优K值
\end{itemize}



