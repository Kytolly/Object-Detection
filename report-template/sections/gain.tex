
本次目标检测实验让我对计算机视觉领域有了更深入的理解,特别是对主流目标检测算法的原理、部署和应用有了实践经验。

通过亲手部署和调用YOLO系列、Faster R-CNN、FCOS等算法的API,
我不仅巩固了这些算法的理论知识,更直观地理解了它们在实际图像处理中的工作方式。
从论文中的抽象概念到代码中的具体实现,这种结合极大地加深了我的学习效果。

使用自行拍摄的校园图像进行实验,
让我认识到实际场景的复杂性。
与标准数据集(如COCO)相比,校园图像可能存在光照不均、
目标遮挡、背景杂乱、小目标密集等问题,
这些都对算法的鲁棒性提出了挑战。

在环境配置、依赖安装和API调用过程中,
我遇到了各种问题,例如CUDA版本不匹配、库冲突、模型加载失败等。
通过查阅文档、搜索解决方案和调试代码,我的问题解决能力得到了显著提升。

